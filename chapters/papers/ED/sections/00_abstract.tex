\begin{abstract}
    
Forests are hotspots of biodiversity, home to over half of the world's vertebrate species, yet are challenging to access and study due to dense foliage and thin branches posing critical challenges for the perception of conventional sensors used on aerial robots. Aerial vehicles flying in such an environment require low-latency and high-resolution depth sensing for obstacle avoidance and would benefit from multispectral sensing for data collection. Current sensors require separate devices for each, depth sensing is coarse and low-latency, and multispectral sensing is also slow and dependent on ambient lighting. We propose a novel system using event spectroscopy, an event-camera based system to reconstruct high-quality and low-latency depth while simultaneously delivering low-latency and high dynamic range multispectral sensing. Depth is reconstructed using structured light, and by changing the wavelength of the projected light, desired bands of multispectral data can be collected. Taking advantage of the low-latency event camera, we iterate through the wavelength of the bandpass filter at a very low latency ($60Hz$) achieving fast spectral data capture for a fixed spatial resolution. Since the same illumination source, a raster scanning laser at $60Hz$, is used for depth reconstruction, both depth and multispectral data can be collected simultaneously. 
We demonstrate that the spectral performance of our system in a lab setting is comparable with off-the-shelf commercial multispectral sensors, and delivers higher resolution depth with increased detection of fine structures when compared with conventional depth sensors. Due to the active projection of light, the system is less reliant on ambient lighting for multispectral sensing, making it suited for use in environments such as tree canopies. A portable version limited to RGB projection is used to collect real-world depth and spectral data from a Masoala Rainforest. Color image reconstruction and the downstream task of material differentiation using this data is then demonstrated. 
This system marks the first application of event cameras for multispectral sensing beyond RGB, and showcases the feasibility of a portable, integrated system which delivers high-resolution and low latency depth and multispectral sensing from a single sensor.


% We show the performance of our event-spectroscope on XX and show that it performs XX better than conventional sensor.
% Additionally, we show the advantage of using depth additionally for material segmentation which results in an improvement over only using spectral data by XX.

% Dense forest canopies are very interesting for observing biodiversity.
% Biodiversity is captured using spectral information. However, current methods lack in two main aspects: (1) low latency (2) low spectral resolution. Additionally, only using spectral information might not be sufficient for material identification in the case of dense forests and therefore depth can be used an an additional modality to improve the performance of material segmentation. 
% To address these issues, we propose an event spectroscope, a novel sensor that enables capture of low-latency, high dynamic range spectral data. We use a full wavelength light source with multiple bandpass filters in combination with an event camera. Taking advantage of the low-latency event camera, we iterate through the wavelength of the bandpass filter at a very low latency ($60Hz$) achieving fast spectral data capture for a fixed spatial resolution.
% The illumination source consists of a raster scanning laser which scans at $60Hz$, establishing a structured light-based depth sensing in combination with an event camera, thus providing depth scans for free.
% We show the performance of our event-spectroscope on XX and show that it performs XX better than conventional sensor.
% Additionally, we show the advantage of using depth additionally for material segmentation which results in an improvement over only using spectral data by XX.
\end{abstract}