\chapter{Conclusion}
\label{ch:conclusion}

\section{Conclusion}

%did a bunch of cool stuff, major contributions

% Recap brief problem statement (necessary?), then contributions%
% Nature in crisis, essential to have data. Direct data, ie collecting dircetly at source enables high temporal/spatial resolution, collection of relevant data types, etc
% essential to
% to enable also end-users to use this data our work focuses on enabling direct data collection from new locations, different data types, 

% Nature in crisis, direct data essential to measure quantify and check the effectiveness of solutions.
% to unlock the full potential of direct data, need to make it accessible to end users, in different locations, and for new data modalities
% The main contributions of our work involve making dirct data more accessible, from different locations, for new data modalities over three main tasks: environmenmtal monitoring in forests, biodiversity monitoring in forests , and pest detection in agrcilucture. 

% VOL & UR |& OG  & BOG & SB & OBR & ED &| EP2 & XP
% for pest detection in agriculuture we demonstrate the collection of plant volatiels using pumps which can be collected and placed using a UAV, enabling scalable early pest detection with data modaoilites. 
% In addition ,we demonstrate a user-centric design for the payload, which enables eaiser collection 
% for environmental monitoring in forests we demonstrate sensor placement using a UAV to enable data collection from the outder tree canopiues through a lighweight helically coiling gripper
% to facilitate easier collection, we demonstarte the same helically coiling gripper composed exclusively out of biodiegradable materials, including a gelatin hydrogel which replacing the elastic which facilitates the coiling with a gelatin hydrogel due to the highe lastic maddulus.
% The same hydrogel can also be used as a bidogredable adhesive for adhesion to different substrates for environmental monitoring tasks. We demonstarte that 0.1g of this adhesive can sustain 20N, the adhesive can adhere to different surfaces of different roughnesses, and finally showcase the versalitityl through three different monitoring tasks, namely placing sensors with a UAV, a perching UAV, and a climbing robot.

% These applications so far have been manually operated, to further enhanhce usability and enable more end-users access to thes etechnologies, they these should be made accessible to as many people as possible. An essential part of this is to automate the workflow, either partially or fully. To achieve this, in OBR we recontruct the depth maps of potentially occluded branches which can enables both navigation in forest environments, and detection of individual branches for sensor placement for instance. Additionally, ED demonstrates the use of event-based structured light for enhanced depth perception as well as multi-spectral sensing.

% For biodiversity monitoring in forests, we look towards surface-based eDNA as a data modality which provides a cumulative snapshop of biodiversity while being scalable to collect. For this we demonstrate manual collection of eDNA from forests in singapore (with esay-to-use interafec ),  as well as autonomous collection in the Amazano Rainforest in brazil. Autonomous collection is enabled through a processing pipeline which first involves mapping the area using RGB images, rapid processing of the data in the field, then using the resuliting heighmap to place sampling points such that the probe maximizes vegitation contact, while maintaining safety for the drone by not going too close to the tree canopy, but close enough to collect data. 

%These works show that there are many ways to collect direct data, using different data modalities, from temperature data, to plant volaties, to eDNA. However, all of these have in common that they require interaction with the environment or objects in the environment. To truly unlock the full potential of direct data, it's collection must be possible in the different locations necesary, at the scales needed, enabled to collect the correct modalities, and espeically usable from the end-users.
%this can be done either through user-friendly hardware design which considers the needs and limitations of the end-users during the design process (UR), easy-to-use interfaces (XP), or through increased automation. Full automation is not required to faciltate use, indirect data collection, such as RGB mapping using UAVs can be a good examlpe of solutions which are easy to use, without requiring complex robotics knowledge. 

%our work marks a first step towards this direction, enabling direct data collection, new data modalities, from more locations, scalably, and makincg this accessible.


\section{Limitations and Outlook} % OK: future research directions

Data collection per area with direct methods is naturally more time and cost intensive than indirect methods, which limits the coverable area with the same effort. 
However covering with direct methods the same area as indirect methods is most likely not needed, and will also result in data processing bottlenecks due to the  high information density. Rather, the high spatial resolution in-depth localized direct data collection methods compliment the indirect methods with large spatial coverage.  Ideally in the future these two data streams would be fused, with a few localized direct measurements which are then extrapolated using the large-scale indirect methods with high spatial coverage to provide detailed, information rich, localized data at scale.
% and direct data which is then extrapolated to scale using indirect methods with high spatial coverage.

%Automation vs teleoperation
%probably full atuomation not needed, look towards direct data as an inspiration (probably incorporate above text also here)

% Robots are made to interact with the environment, for UAVs, this is beginning to be the case, by enabling direct data collection also in environmental use-cases, for instance early pest detcetion in agrciluture, environmental monitoring in forests, or biodiversity monitoring in froests, we hope to contribute and make initial steps to providing actionable data to address the pressing issues facing our natuarl world today.
