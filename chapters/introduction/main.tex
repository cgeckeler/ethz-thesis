\chapter{Introduction}
\label{ch:introduction}

% \dictum[Immanuel Kant]{%
%   Sapere aude! Habe Mut, dich deines eigenen Verstandes zu bedienen! }%
% \vskip 1em

% \begin{otherlanguage}{ngerman}
% Die ältesten Bestimmungen der wahren Grösse der Moleküle hat die kinetische
% Theorie der Gase ermöglicht, während die an Flüssigkeiten beobachteten
% physikalischen Phänomene bis jetzt zur Bestimmung der Molekülgrössen nicht
% gedient haben. \dots
% \end{otherlanguage}

%______________TITLE______________

%Accessible Data Collection from Forests and Agricultural Fields using Aerial Robots through automation and user-centric design

% working title
% Democratizing Scalable Environmental Monitoring, Biodiversity Assessment, and Pest Detection using Aerial Robots

% Accessible, Scalable, and Automatable Direct Data Collection using UAVs in Forest environments and agriculutral fields

%Democratizing Scalable Direct Data Collection from Challenging Environments using Aerial Robots
% 
%_________________________________

%(context: biodiveristy + climate crisis) data collection is necessary to make informed decisions and evaluate effitiveness of measuers [no agri yet, maybe light intro with statistics ]
%Roter Faden: should focus on the points from beginning: some locations are difficult to access, need direct data, need to be scalable, and should be easy to use. 
\section{The Case for Environmental Data} %2-3 pages (actually should be more 1-2)
% why is environmental data needed? => lots of problems/crises, need data to assess
% => 
% - Context: biodiversity + climate crisis, rapid species decliene ,loss of habitat, increase of invasive species. Currently on track to loss a lot of different species.
%seems already plenty of data but most data is actually based on estimates/projections or on single instances from public open source datasets (that are a few decads old) => need up-to-date modern data on actual information (biodiversity, pests, env data)
% also need data to verify that actions taken have the desierd effect (policy makers, but also farmers, smaller shareholders, NGOs, conservation and restoration, 
% data collection is necessary to make informed decisions and to evaluate effectiveness of measures
% also in the context of agriculture, more data has resulted in smart farming practices (ie remote senisng), increasing yield, reducing pesticide use, etc
% to drive informed decisions that optimize efficiency for farming, to optimize land use, reduce crop loss due to pests, reduce pesticides and fertilizer need data. (what data ====>)
%%11-10: after this section, should be clear WHY we need environmental data 


%feb.2023-jan2024
The natural world is in crisis. 
Six of the nine planetary boundaries \cite{Richardson2023} established to measure the stability and resilience of the Earth system have now been crossed, marking the passing of a critical threshold for increasing risks to people and ecosystems.
There is the obvious changing climate, with 2023 marking the first 12 month period to exceed 1.5\degree C above pre-industrial levels \cite{Service2024}, the goal which was set in the 2015 Paris Agreement \cite{Delbeke2019} to avoid or lessen the worst climate harms. Instead, with currently implemented policies, we are on track to reach 3.2\degree C \cite{Report2023} warming by 2100, exceeding even pessimistic projections. The effects can already be felt today, with increasing severity and frequency of extreme weather events, wildfires, droughts, hurricanes, and rising sea levels \cite{Report2023}. 
There are also the direct consequences for the inhabitants of the natural world: mass extinctions and population reductions at unprecedented rates; now dubbed the "sixth mass extinction" \cite{Ceballos2015} or "biological annihilation" \cite{Ceballos2017} with not only entire species, but entire genera being wiped out\cite{Ceballos2023}, driven exclusively through human activity. Biodiversity, one of the most important contributors to ecosystem services, and an essential component of a livable planet for humans, is in decline, driven both directly by anthropogenic pressures with habitat loss due to land use change, and indirectly through climate change. Over the last 50 years, the average size of monitored wildlife populations has shrunk by 73\% \cite{WWF2024}. The current rate of extinction of species is 100-1000 times that of the fossil record \cite{Pimm2014, Ceballos2015}, with an estimated one million plant and animal species threatened with extinction \cite{ipbes2019}.
The drastic effect this has can be seen on land, for example with a net loss of 3 billion birds or 30\% in North America \cite{Rosenberg2019} or obscene shrinking of insect populations (75\% reduction in airborne insect biomass over three decades in Germany, irrespective of habitat type or land use \cite{Hallmann2017}), to the cascading effects insect loss has on ecosystems \cite{Kehoe2021}. Also marine life in the oceans, which are already closer to their thermal limits \cite{Pinsky2019}, face increasing pressure, with over a third of marine mammals, sharks, and reef-forming corals threatened with extinction, and only 3\% of the ocean free from human pressure \cite{ipbes2019}.
% Over 88\% of monitored EU fishstocks are overfished, with the number of fish landed decreasing significantly (94\%) compared to more than a century ago, despite increased fishing technology, newer boats, and increased effort \cite{Thurstan2010}.

Biodiversity loss has even been linked to reducing global terrestrial carbon storage \cite{Weiskopf2024}, exacerbating and amplifying carbon emissions and climate change, as well as cascading effects across ecosystems \cite{Rosenberg2019, Ceballos2015}, affecting human social and economic prospects \cite{Frank2024, Portner2023}.

% I dunno if this transition makes sense, I would rather focus first on the biodiveristy, then this (ie huge decilen, but also lots of undiscovered stuff, should discover. But of the monitored stuff, bias

%macroscopic trends, long-term, can only detect after the fact. (not real time)
%For a more accurate picture, that does not rely on measuring the absence of species as an indicator of biodiversity loss, need up-to-ate scalable, local data. Exisiting data sources are sparse, biased and do not provide the necessary spatial and temporal resolution for assessing the viability of implemented solutions.
Current climate and biodiversity crises \cite{Pereira2024, Weiskopf2024, Pimm2014, Portner2023} underscore the necessity and current lack of comprehensive and scalable environmental and biodiversity monitoring \cite{Gonzalez2023a, McRae2017, Gonzalez2016, Mora2011}. 
Perhaps the clearest such example is the number of threatened species on the IUCN RedList. Of the 163,040 species assessed by the IUCN, almost a third, over 45,000, are deemed to be threatened. Yet the estimated number of species lies over 2 million, meaning that less than 8\% have been assessed \cite{IUCN2024}, all the while undiscovered species face an even higher extinction risk than known ones \cite{Liu2022a}.

Reliance on existing limited data sources perpetuates biases, and hides trends from underrepresented populations, usually from the tropical regions. Proportional weighting indicates steeper and darker trends: global population declines in vertebrate species of 58\% rather than 20\%, for instance\cite{McRae2017}. Moreover, this shows the importance, and current limited availability of, accurate, up-to-date, and representative environmental and biodiversity data. This lack of data hinders our ability to successfully  formulate, implement, and monitor solutions for these problems. 
%TODO LIST WHY data is needed

%Cant understand what you cant measure, currently not enough data (env. + biodiv, need to measure more)

%Figure: why needed? show rates and whwich different ecosystems destroyed/ineffictev methods to recover

% forests stand out for their importance to biodiversity, and also to the level at which these ecosystems are being destroyed/degraded


Forests in particular stand out as biodiversity hotspots, home to over half of the world's vertebrate species \cite{Pillay2022} as well as providing essential ecosystem services and climate regulation \cite{Brockerhoff2017}. They are also under grave threat: deforestation has reached historically unprecedented levels, with one third of primary tropical forests gone \cite{Krogh2021} and 38\% of remaining forests degraded by fire, timber extraction or drought \cite{Lapola2023}.

Forests are usually remote, expansive, tall, cluttered, and thus challenging to access for data collection, even though this data is essential to understand and protect forest functioning and their impact on biodiversity, climate, and overall ecosystem health. This is especially true of tree canopies, which sustain an estimated 40\% of all existing species \cite{Ozanne2003d}, but are most challenging to access. Such data is also vital to evaluate the success of strategies aimed at mitigating climate change and reducing biodiversity loss \cite{Gonzalez2023}, especially within the context of the United Nations (UN) Sustainable Development Goals (SDGs). However, concrete data, especially on biodiversity, ecosystems, and climate are still lacking \cite{Goessmann2023} and costly to collect using conventional, mostly manual methods \cite{Cannon2021, UNEnvironment2019}. 

%RAM paper intro%
% Global agriculture faces unprecedented pressures to feed a growing population, cater to increasingly selective consumer choices, ensure economic viability for the roughly one billion farmers in the world, ensure enough yield with increasingly unpredictable and extreme climate events, and also reduce environmental impact~\cite{McGreevy2022}. Through sustainable intensification (SI), large global change impacts of agriculture can be reduced while increasing yields~\cite{Cassman2020, Pretty2018, Garnett2013}. Many SI strategies leverage productivity gains from increasing biodiversity in farm fields, which can improve land area use and conserve soil through better space-filling. However, the cultivation of such heterogeneous SI fields is more labor-intensive per area than for homogeneous monoculture-based approaches, and not amenable to standard automation strategies for monocultures, which require bare soil and synchronized yields. Smart and adaptive automation is thus critical to upscale SI and provide more food on less land, with fewer negative impacts~\cite{Sparrow2021, Basso2020}.
% A key challenge for SI is the early detection of plant stress for rapid and effective interventions to reduce crop losses, and to optimise the application of pesticides and fertilisers, thereby reducing environmental impact. Precision agriculture thus far has mainly relied on visual imaging technologies such as multi- or hyperspectral cameras to detect plant stress~\cite{Tsouros2019, Toth2016, Singh2020}. A current limitation is that imaging is best suited to identify plant responses to stress at a later stage, i.e., when plants have already undergone substantial damage, because visual damage appears hours or even days after the stress event~(Figure~\ref{fig-1-overview}A)~\cite{mahlein_hyperspectral_2018}.

These changes in the climate and natural world will not only indirectly affect humans through loss of ecosystem services, but also directly, for instance in agriculture, where increases in temperature reduce the global yield of the four main crops constituting two-thirds of human caloric intake: wheat, rice, maize and soybean \cite{Zhao2017}.

%not only forests, also agriculutre, essential for .... and covers over XX land surface area. Essential to keep land productive, yet support or at least limit damage to local biodiveristy. , for instance SI. Need data

Agriculture is essential to feed the world's growing  population and provide livelihoods to the roughly one billion farmers in the world, yet it also accounts for 10-20\% of  global greenhouse gas (GHG) emissions \cite{Gert-JanNabuurs2022}, over 40\% of all land is used for agriculture, 90\% of global deforestation is due to agriculture, as well as 70\% freshwater use \cite{WWF2024}. %add greenhouse gas emssions
The catastrophic decline of bird populations in Europe is also attributed to pesticide use from intensive agricultural land use \cite{Rigal2023}.
This will only increase with rising population sizes, more selective consumer choices, and increasing purchasing power driving increasing meat consumption (60\% of agricultural greenhouse gas emissions are associated with meat, and meat emits 20-30 times GHGs for the same protein content as from plant-based sources\cite{Xu2021, Poore2018}). This necessitates increasing yields while simultaneously reducing environmental impact~\cite{McGreevy2022}.
% need to maximize yield, since more food is needed, but should also reduce environmental impact. possible through SI, but since more effort therefore should minimize crop losses. 
% Through sustainable intensification (SI), large global change impacts of agriculture can be reduced while increasing yields~\cite{Cassman2020, Pretty2018, Garnett2013}. Many SI strategies leverage productivity gains from increasing biodiversity in farm fields, which can improve land area use and conserve soil through better space-filling. However, the cultivation of such heterogeneous SI fields is more labor-intensive per area than for homogeneous monoculture-based approaches, and not amenable to standard automation strategies for monocultures, which require bare soil and synchronized yields. Smart and adaptive automation is thus critical to upscale SI and provide more food on less land, with fewer negative impacts~\cite{Sparrow2021, Basso2020}.
% A key challenge for SI is the early detection of plant stress for rapid and effective interventions to reduce crop losses, and to optimise the application of pesticides and fertilisers, thereby reducing environmental impact. 

One method for reducing the environmental impact of agriculture is through sustainable intensification (SI). The main principles of SI are designed to increase yield and leverage biodiversity to maintain or increase yields while reducing environmental impact by fostering biodiversity, for instance through agroforestry, intercropping or reduced tilling. A central component of SI and also conventional agriculture, is to reduce crop losses due to pests, while limiting the amount of pesticide used. Between 20-40\% of global crop production is lost due to pests \cite{Savary2019}, highlighting the need for early detection. Early detection of infestations can not only reduce crop losses, but also limit the amount of applied pesticides through targeted and early intervention, which will have positive ripple effects on local biodiversity, especially for pollinators and other insect populations. 
%This will need to rely on novel data sources, such as volatiles, since remote sensing detects too late.

The challenge to conserve, restore and preserve natural ecosystems and their function while also maintaining sustainable land use for human use such as for agriculture without detriment to overall biodiversity is hampered by a lack of actionable data. Data either has limited spatial or temporal resolution, due to cost or labor intensive collection, or is limited or missing entirely from areas which are difficult to access, for instance forest canopies. Data collection is usually very manual and labor intensive \cite{Cannon2021}, which does not scale to the dimensions needed to monitor natural ecosystems such as rainforests or fields for agriculture. 

%FIGURE
% Challenges fro environmental data collection: difficult to access locations (picture tropical rainforests/dense
% scalablitiy (large farmlandh)
% ?? data types (satellites data of forests?)



% However, the general decline of biodiversity and crisis in natural ecosystems have spurred governments to action, with the UN passing the SDGs, which beside human-centric issues such as health, poverty, gendter equality ... also places great importance on the natural world, with SDGs such as Life on Land, Life on Water, and XX.

% However, most countries are not on track to meet them, and also data is sparse. [CITE]

% The list of previous statistics can lead to the question whether more data is truely needed. However, scalable, relevant data (at right scales, up-to-date) is needed at all levels (from small shareholder farmer, to NGO, to forest manager, to governments). to make the right decisions. These data are also essential to ensure that the current XX spent/estimated to be needed to address the SDGs are spent in a fashion that actually allows them to achieve their goals. Data is needed to monitor the progress and success of efforts to restore and conserve ecosystems. 

Informative, comprehensive, and up-to-date biodiversity and environmental data is essential to enable informed decisions, monitor progress, and validate the success of implemented strategies to restore and conserve ecosystems and detect pests, at all scales, from smallholder farmers, conservation NGOs, to governmental agencies. 

% where does data come from?
% is it actually hard data or only from some outdated publically available datasets? => estimates usually conservative
% need to monitor results of action, sometimes unintended side consequences, ie carbon credits, but overoptimizing carbon with no side complimentary benefits to biodiversity, thus less effective than it could be 


%Current methods do not allow for full comprehensive picture.  [also include agriculutre]
% - do not collect right data [ie only agri only remote sensing, but volatiles also good.] 
    % - ie indirect/passive observational data
    % - needs direct, interactional data
% - cannot access necessary locations
% - do not cover temporal/spatial scales 
% ie all probles 
\section{The Case for Direct Data}
\label{sec:thesis_case_direct_data}
%There is an insatiable need for data, but in addition to existing data sources, which are mostly indirect and less inforamitve need complementary direct data capture.
% for instance volatiles for agri., collecting env. data directly in situ, or bidoiv data through eDNA
%however 
% more challenging to collect since have to directly capture from the environment: hard to access environments, difficult to scale, challenging for the end-user to use, 
% ==>user-friendliness as a design consideration, robots to enable access to new locations and data modalities, and also open the opiton for automation and scalability.

% - Problem: Current data collection methods are inadequate, do not enable "right" type of data collection, cannot access locations, or do not cover temporal or spatial scales needed for proper data analysis. 
% e.g. mostly remote sensing, also for agriculture. also in forests, mostly use satellite data, in person very difficult access 
% different data modalities are needed for full image (ie challenging to assess biodiversity from satellite images
% collected data types are complimentary to remote sensing
% needs to be automated since pure manual sampling is cost prohibitve ( maybe streamlined, reduced human effort
%
% takeaway=> should collect direct data, ie volatiles, edna, close-up images and sensor data  directly from the source

%%%%%21.11
%% indirect vs direct data (what can be measured with indirect data)
% what is direct data (
% looking at direct data for three applications:
% - environmental monitoring in forests (sensors)
% - biodiversity assessments in forsets (eDNA)
% - pest detections in agriculuture (Volatiles)
% forstes stand out as biodiv hotspots,...... what is eDNA, what are volatiles
% problem is that these methods are not scalable and locations are not easily reached...,

Currently, most scalable data used for environmental or crop monitoring relies on indirect, or passive observation. For agriculture and tree species identification this includes optical imaging technologies such as multi-or hyperspectral, thermal, or light detection and ranging (lidar) data, captured from satellites, aircraft, or UAVs \cite{Tsouros2019, Toth2016, Singh2020, Mahlein2018}. 
These remote sensing methods are clearly very scalable, with large spatial coverage, but with a lower spatial resolution. The further away from the object of interest data is collected, the lower the spatial resolution, and thus also lower relative information density. 
Conversely, collecting data closer to the source results in higher spatial resolution and higher information density, but reduced spatial coverage. For instance, Planet Labs' satellites provide 300 million square km multispectral data at 4m resolution daily \cite{Lab}, whereas UAVs can provide data with cm resolution, but only a few square km.
For applications in precision agriculture, multi- or hyperspectral data can be used to monitor crop health such as water stress as well as detect foliage damage through pathogen infestations \cite{Mahlein2018}. However, for pest detection, crops need to exhibit detectable symptoms, which limits the time of detection until after the pathogen has spread enough to cause observable damage. 
 % While certain species identifications can be made from top-down imagery, such as for trees, this data cannot measure the rich full biodiversity on the ground, especially for smaller animals, insects or plants. 
 
Indirect data for biodiversity is even more limiting, since remote sensing can provide little information on the local biodiversity. Examples include tree species identification from top-down imagery \cite{Onishi2021, Schiefer2020, Xu2020a}, or detecting indicators or proxies for biodiversity, such as leaf cover or vegetation density \cite{Skidmore2015}, which can then be fused with other data. For example, when combining locally collected eDNA data with satellite imagery, estimates on fish population densities and suitability of habitat locations from satellite data can be made\cite{Zong2024}. However, such methods require prerequisite local data, and can only provide a proxy for biodiversity, not the full rich, verified local biodiversity, especially for smaller animals, insects, or plants.
% However, such methods require prerequisite local data, and can only provide estimates, not verified, detailed and grounded local information. However, such approaches can only provide a proxy for the biodiversity, and cannot measure the rich full biodiversity on the ground, especially for smaller animals, insects or plants. 


For environmental monitoring, simple indicators such as land surface temperature can be measured using satellites and provides valuable data on the macro level of potential habitats and living conditions over time, but they cannot capture  subtle local variations, or microclimates \cite{Frenne2021, Zellweger2019AdvancesSensing}.

%FIGURE
% information density vs scalabitiy 
% ie have information density on one axis, and spatial resolution/scalability on the other
%not spatial resolution, since usually resolution is lower than when you are closer

% ideally also have SOTA comparison for access? ie where other solutions exist


These types of information can be collected through direct, interactional data collection. Direct data are localized measurements which require access and interaction with the environment or objects in the environment, for instance physical samples. Examples include, placing sensors at different heights within a tree to measure microclimates, collecting physical samples of DNA for a more comprehensive biodiversity assessment, or close-up image capture with camera traps. In contrast with indirect methods, direct data have increased spatial resolution and information density with reduced spatial coverage. Since the data are collected closer to the source they are necessarily more localized, and have a higher information density. However, since collecting such localized measurements over larger areas becomes time and cost prohibitive, direct data methods are less scalable with reduced spatial coverage. These localized measurements can be taken as often and as close to the object of interest as desired, so  they have a higher temporal and spatial resolution, when compared with the fixed resolution and data capture rate of satellites, for instance. The proximity of direct data collection methods to the data source results in a much higher information density, both in terms of the types of information collected, as well as the level of detail. For instance, more information can be obtained from physical samples of single leaves, to images of the branch or tree taken from a UAV, to the aggregated pixels of forest areas from top-down satellite imagery.

% (direct env. information, individual species information).
Current biodiversity assessments generally follow this direct data approach, either with manual detections (visual inspections in the field, specimen collections, etc) or through more automated methods by placing sensors such as camera traps or audio recorders for bioacoustics \cite{Muller2023}. These methods still require direct access to the field and labor-intensive placement and retrieval of the sensors

The following three use-cases clearly demonstrate the utility of direct data: environmental monitoring in forests, biodiversity assessments in forests, and pest detection in agriculture.


% forest data collection methods
Forests stand out as biodiversity hotspots, but are simultaneously very challenging to access and study \cite{Pillay2022, Ozanne2003d}. This makes previous biodiversity assessment methods difficult to apply and most current direct methods of access are highly manual, and thus expensive and labor intensive. Examples include expensive and invasive fixed infrastructure with limited reach, such as canopy cranes, or unscalable manual labor intensive operations, such as climbers for sample collection or sensor placement, as well as access from below with ladders or telescoping poles \cite{Cannon2021}. 
% More indirect methods include UAV data capture from above, for instance multispectral data.
%Go more into detail about the current manual  different methods, maybe even including the comparison figure from paper.
% direct data collection modalities:
% placed sensors
% close-up multispectral?
% eDNA

%Environmental monitoring: 
Information rich, long term, localized direct data for environmental monitoring can be collected from sensors placed within the region of interest, for instance tree canopies. 
% For environmental monitoring, information rich, long term, localized data can be collected from sensors placed within the region of interest, for instance tree canopies.
% Sensors placed within the region of interest, for instance tree canopies, can provide information rich long term localized data for environmental monitoring. 
However, manually placing and collecting many sensors in environments, such as tall tree canopies is still challenging and has high associated risks and costs.

One additional benefit of direct data collection is the ability to collect novel data modalities that can provide new insights, for example environmental DNA (eDNA) for biodiversity assessments. Existing biodiversity assessments can be invasive, such as with insect traps or tree fogging which collect physical specimens, or are heavily biased towards certain species or animals, such as camera traps or bioacoustic recorders. EDNA is genetic material shed into the environment from organisms in the vicinity, for example through their feces, skin cells, hair, or saliva. These genetic traces can be physically collected through interaction with the environment.  Water eDNA sampling is well established \cite{altermatt-2023, Ruppert2019PastEDNA}, and can present an aggregate view of biodiversity through the aggregation of DNA in watersheds after rainfall. 
% However, this limits the localization of where specific DNA comes from and can overrepresent certain species, such as aquatic species. To get a more accurate snapshot of biodiversity, it is necessary to sample from different environments that may not have water bodies nearby, for instance also forests. 
However, exploring biodiversity in terrestrial ecosystems with limited nearby water bodies is still limited. Exploratory work has demonstrated collecting eDNA from trees \cite{allen-2021, Allen2023}, but such approaches are still exceedingly manual and limited to accessible locations. EDNA presents a noninvasive, informative data modality for detecting overall aggregated biodiversity in different environments, but requires physical samples from the sampling location and scalable collection, especially in hard-to-reach ecosystems remains an open challenge.
% EDNA has shown great promise for detecting overall aggregated biodiversity in a multitude of different environments and ecosystems \cite{}, and present a noninvasive, informative, and automatable data modality for scalable biodiversity assessment in hard-to-reach ecosystems. 


% For environmental monitoring, enabling multispectral data collection within forest canopies can be especially usefull. 

%RAM paper intro
 % However, plants respond to stress within seconds to hours by producing phytohormones and bioactive chemicals from a variety of pathways~\cite{schuman_layers_2016}. Some of these have sufficiently low vapor pressure to be volatile under standard conditions on Earth, and thus can be perceived at a distance. One of the best-studied phenomena in the field of plant-insect interactions is the production of specific plant volatile blends upon herbivory (10's to 100's of compounds) which indicate the timing and nature of the herbivore as well as the severity of attack~\cite{howe_plant_2008, dicke_evolutionary_2010}. Plant volatiles thus serve as important indicators of plant condition prior to visible or severe damage, and also can indicate the success or failure of some SI strategies. Plant volatiles have been used by insects and other organisms over millions of years to forage on plants and assess plant status, but they are still not exploited for precision agriculture~\cite{turlings_tritrophic_2018}. Volatiles represent an interesting alternative or complement to imaging for early and precise detection of stress and subsequent application of mitigation strategies to reduce yield loss. 
Pest detection in agriculture can also benefit from direct data. Currently, precision agriculture relies almost exclusively on indirect remote sensing, such as multispectral data collected with UAVs.
% , but for true early pathogen detection, direct data capture is needed.
For such visual remote sensing to detect damage to the plant, there must be optically observable damage. This occurs hours to days after the stress event, after bulk damage has occurred and potential infestations have spread. However, plants also respond to stress within seconds to hours by producing phytohormones and bioactive chemicals from a variety of pathways~\cite{schuman_layers_2016}, some of which are volatile and can thus be collected in proximity to the plant. These plant volatiles present a unique indicator of both the type and severity of plant stress \cite{howe_plant_2008, dicke_evolutionary_2010}, prior to visible or severe damage to the plant, and thus could present a complementary, direct data solution in precision agriculture for early pest detection. While established approaches exist for sampling volatiles in laboratory conditions, sampling in field conditions is still challenging \cite{lang_ecological_2022,tholl_trends_2021} % 15,16 from RAM
Beyond this, to scale to the dimensions needed for agricultural fields, current manual volatile collection methods are not tenable and automation is essential. 
%In the case of agriculture, precision or smart farming /cite almost exclusively refers to the utilization of additional external data sources in decesion making for pesticited or fertilizer application, for watering, or harvesting times. These data sources usually comprise of remote sensing, either from a UAV flown directly over the field, providing localized, up-to-date, and higher resolution data (multipsectral/thermal), or from commerical satellite feeds which provide coarser information. 
% These data sources are all indirect, since they remotely collect data without direct access to the fields inquestion. This makes it easier to collect data at larger scales, but with less information value compared to direct data sources. 


% Similarly for biodiversity assessment, multiple platforms attempt to provide a silver bullet to the problem of biodiversity monitoring, essentially mainly relying on satellite data, exsisting databases, /cite {gainforest, restor} and sometimes a few localized measurements that are used for extrapolation /cite{that company}. While state of the art algorithms can have impressive performance (e.g. prediciting the habitita of certain fish, inferring the population number of xx). For reliable biodiversity assessments, data still needs to be collected in-situ, whether counting popluations of penguins directly with UAVs /cite, tracking with camera traps, or. ...

% eDNA has emerged in recent years as a promising tool for large-scale biodiversity assessment from single samples. By collecting 
% eDNA intro : collect one sample, aggregated species (water), or manually aggregate through touching many surafecs 
% challenge:difficult to collect

% Environmental data collection also similar, usually just try to extrapolate from a few localized measurements, or use large-scale external (satellite) data to try to combine

% concept of "direct data collection" => ie most current data collections are indirect, passive, and only tangentally measure the features of interest.
% ie remote sensing does not detect the presence of pests, but rather the damage caused by the pests, not direct => need to get close and more directly measure pests
% environmental monitoring proxy measurementns with extrapolations and not in all locations, need to place sensors directly in the region of interest to properly collect data, need to get more into forest and actually maybe interact with forest. 
% biodiveristy monitoring: while large animals (mammals/birds/etc) can be directly seen camera traps, manual visual inspects, etc. other insects/invertebrates/etc harder to detect and accurately count. => EDNA allows direct physical proof of existance, but also requiers physical interaction


% remote data is great, since localized, rich in information, but a) more time/consuming to collect ie not scalable, b) challenging to collect depending on location (forests, agriculutural fields)   => robots?

%summary

% and automatable data collection for new data modalities such as volatiles or eDNA. These showcase the need for automatable robots to assist with data collection.
 % Complimentary data sources also needed. 
 % Direct data collection methods deliver informative and localized measurements, but face challenges for scalability, accessing certain locations, and reducing collection effort for new direct data modalities such as eDNA or plant volatiles.
Direct data can deliver localized and highly informative data, however, there remain open challenges.
Major open challenges for current direct data collection methods are accessibility from certain environments and locations, scalability to reduce the high manual effort needed for collection, and collecting the relevant data for a task by exploring new data modalities such as eDNA or plant volatiles.
Currently, direct data collection are largely manual, which not only limits the coverable area, but also location accessibility. Especially for new data modalities such as eDNA or plant volatiles, the collection effort remains high. These methods do not scale to the spatial and temporal dimensions needed for agriculture or to monitor expansive and remote ecosystems such as forests. These challenges invite the use of robots to enhance accessibility and automation to improve scalability while simultaneously enabling the collection of new data modalities. %reaching new locations. 
 %Problem: access, scalability, data modalities

% current state of robots for direct data collection 
% IE SOTA here
% maybe different sections for biodiv/env/agri
% robots provide opportunity to solve these issues, collecting complementary data modalities, scaling to sarger spatial/temporal, and providing access to more locations (also have the research questions later, so don't go too much in depth)
% ie robots can enable scalable direct data collection from previously inaccessible locations
\section{Robots for Data}
%*% Direct data collection does not scale, some locations difficult to access, also more challenging from control for users to do indrect data collection
% => use robots! show robots already being used for direct data collection (ie senor placement, etc, etc. SOTA)
% manual edna, volatiles, limited location access (ie sensor placement not on outer canopy
% ==>user-friendliness as a design consideration, robots to enable access to new locations and data modalities, and also open the opiton for automation and scalability.
% robots can enable direct data collection.
% using aerial robots can also enable access to challenging locations
%user friendliness as a design consideration to facilitate end-user adoption
%since robots, can also automate and scale 
% gradient of complete manual control (teleoperation) to complete automation
% still currently never full manual control, always some offloaded ie control, PID, , or then waypoint missions maybe abstacle avoidance to full autonomy. 

%%%%%%%%%%%%%%%%%%%21.11 
% indrect data easy to collect
% Direct data challenging to ocllect
%% % this is problem since end-users are xxx
    % can therefore augment with automation or user-friendliness
%% current solutions are XXX
%% main challenges are still xx
%this is then what .we address (thesis outline)



%FIGURE:
% collage with different solutions, ie similar to snailbot, ie sensor placement, perching, other data collection

%new intro
% robots have been used extensively for data collection, also in these enevironments, but mostly for indirect data (commerically). A few direct data coputre methods exist it literature (e.g. sensor,deleaves, etc.)
% challenge/limitation of previous approaches is still the limitation of accessing some areas (e.g. agriculutural fields, outer canopy)
% can be challenging to collect novel data types (volatiles, edna),
%but most of all to make it user friendly.


% Figure: forest access methods (ie climber, crane, etc) and currently inaccessible locations?

While direct data collection provides increased information richness, it is also more challenging to collect, especially at scale. 
This motivates the use of robots, in particular aerial robots or UAVs, which can easily access remote and challenging environments. Coupled with automation, direct data collection can thus be scaled, also from previously inaccessible locations.
Indeed, robots have been and are currently being used extensively for indirect data collection in challenging environments; for example, UAVs for mapping, lidar and multispectral data capture over forests. They now represent mature, cost-effective platforms that are easy to use out-of-the-box, thanks to advanced algorithms and sensing capabilities which abstract away low level control for the users, providing stabilization, obstacle avoidance, waypoint navigation, and in some cases even target tracking and trajectory generation. This results in tools with which almost anyone can collect data with minimal training or introductory overhead. 

% this is problem since end-users are xxx

%% current solutions are XXX

%% main challenges are still xx

%this is then what .we address


Direct data collection in forests are also seeing increased use of UAVs.
% Increasingly, UAVs are being used for direct data collection in forest environments. 
Examples include collecting physical samples \cite{Krasylenko2023}, such as sampling small branches from the tops of trees \cite{Charron2020}, or collecting endangered plants from the edges of cliffs \cite{LaVigne2022}, placing sensors \cite{Hamaza2020, Farinha2020}, sampling eDNA from branches \cite{Aucone2023a}, as well as close-range visual data collection \cite{Liu2022, Zhou2022}.

However, there still exist three main challenges when considering direct data collection with robots: accessibility, new data modalities, and user-friendliness.

Certain environments and areas remain inaccessible and unexplored for direct data collection, robotic direct data collection for new data modalities such as plant volatiles or eDNA is limited, and lastly all developed solutions should be easily usable by the end-user so that developed solutions are actually adopted for use in the field.
% certain areas remain inaccessible and unexplored, 
% currently most data is of a single type and it is unexplored to collect complimentary data types that can provide complimentary or more relevant data,
% direct robotic data collection of new data modalities complimentary to exsiting ones is unexplored.
% robotic collection of novel data modalities such as eDNA or plant volatiles is unexplored,
% and lastly how to make all of this user friendly so that these solutions can also be adopted in the field. 

%###new locations
Unexplored regions for direct data collection include the thin branches of the outer canopy for collecting sensor data, for instance, despite this region being of particular interest since it is the site of important photosynthetic activity, gas exchange, and water transpiration. 
Similarly, tree canopies sustain an estimated 40\% of all existing species, but access the inside of the canopy for biodiversity monitoring remains exceptionally challenging due to the dense and cluttered foliage. %[eprobe, snailbot]
Lastly, direct data collection from agricultural fields through placement and collection of sensors is also largely unexplored, with most approaches in precision agriculture focusing on indirect data collection from above \cite{}, or spraying crops from above with UAVs\cite{}.
% agri field remain unexplored for placing and collecting sensors, most approaches focus and NDIV data collection, or sparing applications, not directly interacting with the environment by placing sensor.
%%%%%%%%%%%%%%%%%%%%%maybe move this to the case for direct data?, 
%### data modalities
% new data modalities

Direct data collection methods also enable the collection of different and novel data types which can be more informative, for instance eDNA or plant volatiles.


% There are several direct data types which can provide more informative data on the environment. For instance, by placing sensors within the zone of interest, microclimates (also vertical gradients) can be more easily measured, without requiring climbers, fixed infrastructure or other challenging and expensive methods of attachment. In agriculutural fields, remote sensing for precision agriculture still considers indicies such as NDVI or other optical based methods a way of monitoring plant health and there is hope that by expanding to different wavelengths of the electromagnetic spectrum, damage to the leaves can still be detected eralier, at least before discernible by the naked eye. This is limited by two fundamental limitations, for pest infestations to be detected, there has to be optically detectable damage to the plant leaves, and data collection is always taken a distance away from the plant, meaning that a pixel of the data capture usually averages across several leaves, several plants, or several meters of crops, depending on the distance away of the data capture. This means that even more damage must be present before it will be detecetd in the averaged data. 
% - compared to volatiels, which are released immediately.
% - biodiveristy assessments, edna well suited since non-invasive, aggregating, and can capture fuller picture of biodiversity vs convential (biased methods) ie bioacoustics, camear traps

%instead, here rjust say:
% As mentioned in Section~\ref{sec:thesis_case_direct_data}
% For forests, direct data methods such as direct sensor measurements from the region of interest (different strata of the tree) can provide rich informaiton,
For agriculture, volatiles can present an early pest detection method which can detect damage almost as soon as it is inflicted compared with other methods which require visible damage. For biodiversity monitoring, eDNA presents a noninvasive aggregated method which can capture a fuller picture of the biodiversity in an area than existing conventional methods.

However, collecting these data modalities through robotic methods remains largely unexplored. Both to access certain environments, and to scale these methods to dimensions needed for productivity, robotic data collection methods are indispensable. 


% it agriculutre mostly terrestrial robots for weeding and weed detection, some commercial aerial applications for spraying and of course remote sensing. Sensor placinig in agriculuture not well explored, could also be used in other fields, such as the distribution of phereomne traps for invasive species (Japan beetle, wineyard bug). % volatile collection under explored in field conditions, especially for UAV deployment and collection of sensors

In general, robotic collection methods open possibilities to access new locations, capture data at specific and more frequent times, as well as partially or fully automate to save time and expand access.
% however points xyz have not yet been addressed, which is what I add

% % teleoperation vs automation ( ideally automate everything, however unlikely, so should make available to users
% % end-user not necessarly professional drone pilot, but other scientists, so should make accessible. 
% % maybe end goal is not full automation, but partial automation with human-in-the-loop, ie all the difficult things are automated so many people can access and do, but the decisions and final say is still with the human pilot.
% % => easier legally, pepole also mer accepting, and doesn't require as much complete redundancy as fully automating everything. Also current state of robotics/ML not suited for full auotmation 
% % still have increased efficiency and reduced human labor since now one person can supervise multiple robots and tasks are performed faster.

% In the case of volatiles for agriculture, while direct data capture is more informative, it is also more timeconsuming and can cover less larger scales than traditoinal remote sensing methods. To still enable use at the scales needed for agriculture, robots are needed.
% % aerial robots enable covering of large areas, and through the option of automation can also image larger scales. 

% => UAVs as a way to access. eg..g sensor deployment, perching, etc.
% However, still need specialized hardware, control, complex (ie de-leaves dual person collection)

%  goal/contributions
% \subsection{Automation and Teleoperation} % maybe just have a section without explicitly making 消食 
%NOW end user is not necessarily robot expert, so should use easy integrated hardware, and simlpe-to-use software (for tele), but also atuomate as mucha s possible since this makes it also easier to use
%*% in order to make solution user-friendly and allow end-users to easily use the solutions, should either automate or incorporate user-friendly design
% provide previous points, but accessible to larger user base. IE collect new data modalities, enable access to new locations, scalability, but also accessible to larger user base through automation aund enhanced (user-friendly) design and teleoperation
%also brief SOTA that full automation is unrealistic in natural environments in the near future, and enabling new and scaalble applications with current technologies has larger impact.

%## USer friendliness

Robots, in particular aerial robots or UAVs, can provide access to previously challenging areas, as well as scalability through data capture at specific and more frequent times with partial or full automation to save time and expand access. In any case, however, humans will need interact with the system. Either for conducting the operation, sample collection, or data analysis afterwards. In order to ensure availability of direct data through adoption of these technologies, the end-user must be able to easily use the data collection methods.


Since direct data capture methods are inherently more risky and challenging to conduct than indirect data capture due to the proximity of the environment and the interaction with objects therein, it is vital that automation and user-friendliness are at the forefront, already during the mechanism and payload design to ensure that the end-users will be able and willing to adopt the solutions in the field.

Indirect passive data collection methods provide a good inspiration, UAVs for photography are now very simple to use. Permits and legal issues aside, these UAVs can be  flown by most people with no previous knowledge within a few minutes. This is due to a mix of automation and user-friendly design. Through sensor integration and intelligent algorithms, low level control is abstracted away, leaving the user to provide simple high level commands for flight direction. Many commercial UAVs also have obstacle avoidance sensors, which further facilitates usability by removing the need for the user to also monitor obstacles and ensure collision-free flight. Supervised autonomy, such as waypoint based flight for photogrammetry collection resembles the simplest use-case, the user programs the flight path and should monitor the progress, barring any errors, in practice this can be done fully autonomously, also beyond range of the remote.


Despite increased attention to robotic direct data collection methods, they are unfortunately not yet at the level of readiness compared with the out-of-the-box ready to use indirect methods such as UAV mapping.
% robotic direct data collections methods are also seeing increased attention, but, unfortunately, are not yet at this level of readiness. 
Part of the rapid spread and adoption of UAV technologies, for instance, for mapping is their ease of use. For direct data collection methods to also be a viable data collection tool in the field, they also need to present enhanced usability, especially since increasingly the end-users in the field are not robotic experts eager to debug issues in the field, but ecologists, biologists, conservationists, and other professions which would benefit from the data. There are two main methods to democratize availability of direct data collection methods, automation and enhanced teleoperation. By automating at least certain parts of the data collection procedures, the user can be unburdened and focus on more essential tasks, such as sample site selection. Even with teleoperation, the user can be assisted by considering the requirements and limitations of the user during the design stage, implementing enhanced user-friendliness during the development of direct data collection method.

This level of usability should also be attained for direct data collection methods to ensure widespread adoption and to allow the people who need to use it to use it. Similar to the indirect case, this involves a mix of user-friendly design and automation. Incorporating user-friendliness already at the design stage can enable payloads to become easier to use when teleoperated, and through different levels of automation the solution can also become easier to use. 

\subsection{Research Objectives}

Direct data can provide more informative data, with higher spatial and temporal resolution, as well as new data modalities. Examples include environmental monitoring of forests with highly localized data collection, eDNA for biodiversity monitoring in forests, or collecting plant volatiles in agricultural fields for early pest detection. 

However, direct data collection methods are more challenging to scale since these collection methods require more effort to collect, and the proximity to the data source makes collection from certain locations such as tree canopies more challenging, as well as scalability such as large agricultural fields. Robots, especially aerial robots can be a part of the solution, but the interactional nature of data collection makes robot control more challenging and dangerous. Therefore end-user friendliness is also essential. 

%double-cehck what is research objective
This leads to the main research objective of this dissertation: to democratize scalable direct data collection methods for environmental monitoring, biodiversity assessment, and agriculture in new locations and for more users. This involves enabling novel use-cases for UAVs to collect direct data from previously unexplored areas, such as the outer or inner tree canopies, or agricultural fields, enabling collection of novel types of direct data which can provide more informational value, such as eDNA or volatiles, and taking steps through automation and user-friendly design to ensure that end-users can utilize the developed tools. 

Concretely, this translates to following four research questions, which we will investigate in this dissertation within the context of pest detection in agriculture, and biodiversity and environmental monitoring within forests:

% 1. ie enable new types of data collection, requiring direct interaction with the environment or objects in the environment
% 2. enable more people to use collect this data through automation and user-friendliness
% - enable new novel locations through the use of UAVs. 

% [maybe this comes after the research questions, as the answer why they are needed]
% - Contributions:
% 1. Collect novel (relevant, vs remote sensing) types of data of environmental factors (eDNA, volatiles) [new data modalities]
% 2. Initial steps to provide easy-to-use solutions for end-users through automation and user-centric design  (user-friendliness, new data modalities)
% 3. Enable novel use cases for UAVs, placing and collecting sensors from tree canopies, or placing and collecting sensors from agricultural fields (new locations, new data modalities)
% 4. Provides access to previously inaccessible locations, and the use of automatable drones permits scaling to larger scales (scalable, new locations)


% uniquely challenging to enable new direct data colleciton methedosd, adderssing issues ,therefore ask:
%Democratizing scalable data collection methods for environmental monitoring, biodiversity assessment, and agriculture for all locations and all users. (more locations and more users
%
% In this dissertation we look at the following four research questions within the context of pest detection in agriculture, and biodiversity and environmental monitoring within forests.

%FIGURE?
% a nice way of presenting the different works of thesis WRT to access/what they enable?
% biodiv assessment, env. monitoring, agriculture (columns/different envi. ie animals in forest, then just forest, then agri field
% x axis is scalabilty/ automation + user-friendliness?

%bit more introduction on the research questiotns, maybe one central one?
% Thesis outline /research questions
% Q1. What type of data is necessary to collect for holistic view of environment/biodiv/pest (direct data collection)
% Q2. How to democratize direct data collection
% Q3: how to enable direct data collection from previously inaccessible (difficult to access) locations?
% Q4: How to scale?
% => (summarized)
% Q1: What data is needed for direct data collection? [eDNA, volatiles, direct sensor data]
Q1: What direct data is needed for environmental monitoring and biodiversity assessments in forests and pest detection in agriculture?\\ % [ eDNA, direct sensor data/spectral(?), volatiles]
% what direct data types provide the most informative data for an environment and task, and how to collect it?
	%alternative: How to collect direct data for different fields (agriculture, biodiv/env.)
Q2: How to democratize direct data collection?\\ %[automation, user-friendliness]
Q3: How to scale direct data collection?\\% [use UAV to access new locations + automate ]
Q4: How to collect direct data from challenging locations?\\% [UAVs for new locations]



\section{Dissertation Outline}

\begin{table}
    \centering 
    \newcolumntype{L}{>{\raggedright\arraybackslash}X}
    \begin{tabularx}{\linewidth}{L|cc|ccccc|cc}
            % VOL & UR & OG  & BOG & SB & OBR & ED & EP2 & XP
        & \multicolumn{2}{c}{Pest D.} & \multicolumn{5}{c}{Env. M.} & \multicolumn{2}{c}{Bio. A.}  \\
         % & Pest Det. & Pest Det. & Env. Mon. & Env. Mon. & Env. Mon. & Env. Mon. & Env. Mon. & Biodiv. Assess. & Biodiv. Assess. \\%TODO: make multicolumn for each of the papers
         & \cite{Geckeler2023a} & \cite{Geckeler2024a} &  \cite{Geckeler2022a} & \cite{Geckeler2023b} & SB & \cite{Geckeler2024} & ED & \cite{Kirchgeorg2024} & XP \\
         \hline \hline
         New Data Types (RQ1)  & X &   &   &   &   &   & X & X & X\\
         % maybe novel data modalities, novel direct data types
         \hline
         Automation (RQ2)           &   & X &   &   &   & X & X &   & X\\
         \hline
         Scalability (RQ3)          & X & X &   &   & X &   & X &  X & X\\
         \hline
         User Friendliness (RQ2)    &   & X &   & X &   & X &   &  X & X\\
         \hline
         New Locations (RQ4)        & X &   & X & X & X &   &   &  X & X\\
    \end{tabularx}
    \caption{Papers included in this dissertation grouped according to main themes: pest detection in agriculture (Pest D.), environmental monitoring in forests (Env. M.), and biodiversity assessments in forests (Bio. A.), as well as which research questions (RQ1-4) are addressed in each.}
    \label{tab:paper_overview}
\end{table}

Table~\ref{tab:paper_overview} gives an overview of the papers included in this thesis and shows their the contributions to each of the four research questions (RQ1-4).The papers are grouped according to the application area, starting with pest detection in agriculture (Pest D.), environmental monitoring in forests (Env. M.), and lastly biodiversity assessments in forests (Bio. A.).

The following section contains a brief summary of every work and its relevance to the research questions. Next, Each the full works for each section are listed.

The rest dissertation is structured as follows: first, there is an overview of works and other relevant activities. Next, the there are three main sections each containing the main application area: pest detection in agriculture, environmental monitoring in forests, and biodiversity monitoring in forests. Nine chapters are are included within the three main parts.

Finally, in the discussion we summarize the contributions of our works and how they address the research questions, as well as limitations and open research areas. %future research directionsl/open research areas 


\subsection{Pest Detection in Agricultural Fields}
In these works, we investigate early pest detection for agriculture using plant volatiles. By placing and collecting samplers using a UAV we are able to successfully capture the plant volatiles (RQ1), as well as enabling scalable sensor placement for agricultural fields (RQ3,4). User-friendliness is addressed with a new user-centric payload design, enabling the users to place and collect sensors within minutes of practice using an off-the-shelf UAV with only a simple mechanical attachment (RQ2).
% After both papers, have short section describing next steps (ie also have multiple pumps autonomous deplomyent and collection 
% multiple pumps?/automation?
% Part A - looking at XXX
% Paper x,y,z which did whatever...

%exectuive summary, what we do, what research questions we look at, what we achieve
% ie enable collection of new data type volatiles with UAVs from agricultural fields, (Q1,Q3,Q4), also look at user-friendliness, scalability and automaiton

\subsubsection{Robotic Volatile Sampling for Early Detection of Plant Stress: Precision Agriculture Beyond Visual Remote Sensing \cite{Geckeler2023a}}
%Abstract%
% Global agriculture is challenged to provide food for a human population that is larger than ever before and still increasing. This is accompanied by the need to reduce the large global impacts of agriculture while increasing yields. Early identification of plant stress enables fast intervention to limit crop losses and optimized application of pesticides and fertilizer to reduce environmental impacts. Current image-based approaches identify plant stress responses hours or days after the stress event, usually only after substantial damage has occurred and visual cues become apparent. In contrast, plant volatiles are released seconds to hours after stress events and can quickly indicate both the type and severity of stress. An automatable and nondisruptive sampling method is needed to enable the use of plant volatiles for monitoring plant stress in precision agriculture. In this work, we detail the development of a plant volatile sampler that can be deployed and collected with an uncrewed aerial vehicle (UAV). The effect of sam- pling flow rate, horizontal distance to volatile source, and overhead downwash on collected volatiles is investigated, along with the deployment accuracy and retrieval successes with manual flight. Finally, volatile sampling is validated in outdoor tests. The possibility of robotic collection of plant volatiles is a first and important step toward the use of chemical signals for early stress detection and opens up new avenues for precision agriculture beyond visual remote sensing.

In this work we investigate the process of robotic volatile collection for early pest detection in agriculture. In contrast to remote sensing, which requires visual symptoms and thus damage on the plant to detect pests, volatiles are released seconds to hours after the herbivore event. Volatiles can thus quickly indicate both the type and severity of stress. An automatable and nondisruptive sampling method is needed to enable the use of plant volatiles for monitoring plant stress in precision agriculture. Therefore, we develop a plant volatile sampler that can be deployed and collected with UAV. The effect of sampling flow rate, horizontal distance to volatile source, and overhead downwash on collected volatiles is investigated, along with the deployment accuracy and retrieval successes with manual flight. Finally, volatile sampling is validated in outdoor tests using real plants and simulated herbivory. The possibility of robotic collection of plant volatiles is a first and important step toward the use of chemical signals for early stress detection and opens up new avenues for precision agriculture beyond visual remote sensing.
The collection of direct data from agricultural fields, in this case volatiles, allows more informative data collection than from indirect methods, such as remote sensing. In this case, the increased information is temporal, with direct data permitting earlier detection of pests. 
% However, as with most direct data collection methods, data collection is also more work, requiring the placing and collecting of samplers from the field, which is more challenging than simply flying above it. 
This work therefore addresses RQ1, enabling the direct data collection of a new data modality, namely plant volatiles; this work also addresses RQ4, enabling the placing and collecting of sensors in agricultural fields, a previously underexplored location for sensor placement. 
%new modality/new locations


\subsubsection{User-Centric Payload Design and Usability Testing for Agricultural Sensor Placement and Retrieval using Off-the-Shelf Micro Aerial Vehicles \cite{Geckeler2024a}}
%Abstract
%—Increased flight time and advanced sensors are making Micro Aerial Vehicles (MAVs) easier to use, facilitating their widespread adoption in fields such as precision agriculture or environmental monitoring. However, current applications are limited mainly to passive visual observation from far above; to enable the next generation of aerial robot applications, MAVs must begin to directly physically interact with objects in the environment, such as placing and collecting sensors. Enabling these applications for a wide spectrum of end-users is only possible if the mechanism is safe and easy to use, without overburdening the user with complex integration, complicated control, or overwhelming and convoluted feedback. To this end we propose a self-sufficient passive payload system to enable both the deployment and retrieval of sensors for agri- culture. This mechanism can be simply mechanically attached to a commercial, off-the-shelf MAV, without requiring further electrical or software integration. The user-centric design and mechanical intelligence of the system facilitates ease of use through simplified control with targeted perceptual feedback. The usability of the system is validated quantitatively and qualitatively in a user study demonstrating sensor deployment and collection. All participants were able to deploy and collect at least four sensors both within 10 minutes in visual line-of- sight and within 12 minutes in beyond visual line-of-sight, after only three minutes of practice. Enabling MAVs to physically interact with their environment will usher in the next stage of MAV utility and applications. Complex tasks, such as sensor deployment and retrieval, can be realized relatively simply, by relying on a mechanically passive system designed with the user in mind, these payloads can enable such applications to be more widely available and inclusive to end-users.

Building on the previous work we now address RQ2 and RQ3, enabling more user-friendly collection and retrieval of the sampling mechanism, which also leads to more scalable data collection and lays the initial framework for automation. Many of the user-cues and mechanical intelligence that make this mechanism easier for the user to use, also make it easier to automate, since both humans and algorithms benefit from clear, detectable transition criteria and robust hardware.  Taking inspiration from commercially available mapping UAVs, where most of the challenging and monotonous tasks are automated, and thus they require little specialized training to use; we seek to enable new aerial robotic applications, such as sensor placement and collection. By considering the end-user already during payload development, these tasks can be achieved using simple control schemes with off-the-shelf UAVs. Enabling these applications for a wide spectrum of end-users is only possible if the mechanism is safe and easy to use, without overburdening the user with complex integration, complicated control, or overwhelming and convoluted feedback. To this end we propose a self-sufficient passive payload system which enables both the deployment and retrieval of volatile sensors for agriculture. This mechanism requires only a simple mechanical attachment to a commercial, off-the-shelf UAV, without further electrical or software integration. The user-centric design and mechanical intelligence of the system facilitates ease of use through simplified control with targeted perceptual feedback. The usability of the system is validated quantitatively and qualitatively in a user study demonstrating sensor deployment and collection. All participants were able to deploy and collect at least four sensors both within 10 minutes in visual line-of- sight and within 12 minutes in beyond visual line-of-sight, after only three minutes of practice. Enabling MAVs to physically interact with their environment will usher in the next stage of UAV utility and applications. Complex tasks, such as sensor deployment and retrieval, can be realized relatively simply, by relying on a mechanically passive system designed with the user in mind, these payloads can enable such applications to be more widely available and inclusive to end-users.

%scalability, automation, user friendliness


\subsection{Environmental Monitoring in Forests}
Forests represent a vital biosphere, covering a third of terrestrial land area \cite{FAO2020a} and home to half of all vertebrate species \cite{Pillay2022}.
However, they are challenging to access, with dense and cluttered foliage posing a challenge for robots, both for accomplishing tasks such as sensor placement as well as for perception and navigation. First, we enable data collection from the thin branches of the outer canopy (RQ4) using a bistable UAV-deployed gripper. The gripper is lightweight, weighing less than 5g, but can still support 280g across different branch diameters. Since sensor collection can be challenging, a fully biodegradable version of this gripper, comprised entirely of non-fossil based, biodegradable components is developed. This gripper coils and is actuated with a water-soluble gelatin-hydrogel, thus it will uncoil and can be easily collected after rainfall facilitating sustainability and usability (RQ2). The same hydrogel is then shown to also function as an adhesive, through heating, pressing against the surface, and cooling. This enables robotic monitoring tasks requiring adhesion across various substrates, such as placing sensors, a perching drone or a climbing robot. In the forest, this can enable access to the different forest strata; sensors can be placed against the trunk fo the tree in the understory, the inner canopy can be reached with a climbing robot, and the emergent layer can be monitoring with a perching UAV (RQ3,4).

User-friendliness can can be enhanced full or partial automation, in the last two works we look at perception strategies which can be used as building blocks to enable UAV data collection and navigation in forests. First, we reconstruct depth maps of potentially occluded branches, thus removing foliage and leaving the tree skeleton. Using this data, UAVs can avoid fatal collisions by avoiding thick branches while pushing aside foliage, or detect suitable branches for sensor placement. Next, we demonstrate high-fidelity depth and multispectral sensing utilizing an event-camera structured light setup. Multispectral sensing is achieved by projecting different wavelengths as structured light, this multispectral sensing is less dependent on ambient light and can thus be collected closer to the point of interest, such as forest canopies, presenting a new type of multispectral data collection (RQ1). The depth and multispectral sensing is showcased on the downstream task of material differentiation, detecting leaves and branches. The low latency, high-framerate of the event camera makes the setup uniquely suited for dynamic, rapidly changing environments, such as a UAV flying through a forest. Both methods can be used to assist navigation of UAVs in forest environments, and thus can also assist with automation (RQ2), scalability (RQ3), as well as user-friendliness (RQ2).

% in these chapters look to develop scalable, user-friedly, autoamatable solutions to collect direct data from previously inaccessible locations.

\underline{Attachment for Data Collection}

\subsubsection{Bistable Helical Origami Gripper for Sensor Placement on Branches \cite{Geckeler2022a}}
%abstract
%Understanding forest functioning is limited by the scalability of monitoring solutions and difficulty of access. Manual sensor placement can reach most locations but lacks scalability. Micro-aerial vehicles (MAVs) allow for scalable sensor delivery, but current solutions are limited to attaching sensors to the trunk or large branches with spines or adhesives. The thinner branches of the outer canopy remain inaccessible, despite being of particular interest due to the important physiological processes occurring in the foliage. Herein, a MAV-deployable bistable helically coiling origami gripper is developed. The unfurled state allows for transport with a MAV, and when pushed against a branch triggers the second helically coiled state, which permits secure attachment to branches. Origami manufacturing keeps the weight of the gripper below 5 g, despite holding up to 280 g, and gripping diameters from 8mm to 38 mm inclined up to 30°. The holding force, activation force, and resistance to tilt and rotation offsets are experimentally characterized. The deployment and retrieval of the gripper and sensor are demonstrated outside, where sensor data are collected from previously inaccessible branches in the outer canopy. Enabling robust sensor
%attachment in the outer canopy marks a step toward scalable environmental monitoring of forest ecosystems.

In this work we look at RQ4, enabling direct data collection from the previously inaccessible outer tree canopy in forests through sensor placement and collection using UAVs. Manual sensor placement can reach most locations but lacks scalability. UAVs can enable scalable sensor delivery, but current solutions are limited to attaching sensors to the trunk or large branches with spines or adhesives. The thinner branches of the outer canopy remain inaccessible, despite being of particular interest due to the important physiological processes occurring in the foliage. In this work UAV-deployable bistable helically coiling origami gripper is developed. The unfurled state allows for transport with a UAV, and when pushed against a branch triggers the second helically coiled state, which permits secure attachment to branches. Origami manufacturing keeps the weight of the gripper below 5 g, despite holding up to 280 g, and gripping diameters from 8mm to 38 mm inclined up to 30°. The holding force, activation force, and resistance to tilt and rotation offsets are experimentally characterized. The deployment and retrieval of the gripper and sensor are demonstrated outside, where sensor data are collected from previously inaccessible branches in the outer canopy. Enabling robust sensor
attachment in the outer canopy marks a step toward scalable environmental monitoring of forest ecosystems.


\subsubsection{Biodegradable Origami Gripper Actuated with Gelatin Hydrogel for Aerial Sensor Attachment to Tree Branches \cite{Geckeler2023b}}
%abstract
%Forest canopies are vital ecosystems, but remain understudied due to difficult access. Forests could be monitored with a network of biodegradable sensors that break down into environmentally friendly substances at the end of their life. As a first step in this direction, this paper details the development of a biodegradable origami gripper to attach conventional sensors to branches, deployable with an aerial robot. Through exposure to sufficient moisture the gripper loses contractile force, dropping the sensor to the ground for easier collection. The origami design of the gripper as well as biodegradable materials selection is detailed, allowing for further extensions utilizing biodegradable origami. Both the gripper and the gelatin hydrogel used as an actuating elastic element for generating the grasping force are experimentally characterized, with the gripper demonstrating a maximum holding force of 1 N. Additionally, the degradation of the grip- per until failure in the presence of moisture is also investigated, where the gripper can absorb up to 10 ml of water before falling off a branch. Finally, deployment of the gripper on a tree branch with an aerial robot is demonstrated. Overall, the biodegradable origami gripper represents a first step towards a more scalable and environmentally sustainable approach for ecosystem monitoring.

The sensor collection of the previous work can be challenging, therefore in this work we develop a fully biodegradable version of the gripper composed entirely of fully biodegradable, non fossil-based materials. The gripper coils and is actuated with a water-soluble gelatin-hydrogel. Through exposure to sufficient moisture the gripper loses contractile force, dropping the sensor to the ground for easier collection. This allows easy collection from the forest floor after rainfall facilitating usability (RQ2). 
The origami design of the gripper as well as biodegradable materials selection is detailed, allowing for further extensions utilizing biodegradable origami. Both the gripper and the gelatin hydrogel used as an actuating elastic element for generating the grasping force are experimentally characterized, with the gripper demonstrating a maximum holding force of 1 N. Additionally, the degradation of the grip- per until failure in the presence of moisture is also investigated, where the gripper can absorb up to 10 ml of water before falling off a branch. Deployment of the gripper on a tree branch with an aerial robot is demonstrated. Overall, the biodegradable origami gripper represents a first step towards a more scalable and environmentally sustainable approach for ecosystem monitoring.

\subsubsection{Robotic Environmental Monitoring using Gelatin Hydrogels as a Biodegradable Adhesive}
%abstract
%Scalable data collection from challenging locations, such as forests or bridges is essential for biodiversity and environmental monitoring, as well as infrastructure and industrial inspection. Robots can collect this data, by placing sensors with uncrewed aerial vehicles (UAVs), perching UAVs, or climbing robots. All require adhesion to substrates with varying roughnesses, from tree bark to concrete and glass. Unfortunately, common adhesion methods are specialized for specific substrates, don't generalize to different surfaces, or leave behind harmful residue. This work presents the novel use of gelatin-based hydrogels as biodegradable and water-soluble adhesives for reversible adhesion to different surfaces for robotic environmental monitoring. The hydrogel adheres through heating, attaching, and cooling. The hydrogel is released by heating again, and any residue can be washed away with water. To correctly dimension the adhesive, factors affecting the maximum pull-off force are experimentally characterized. The versatility of the adhesive is shown through adhesion to different surfaces. Only 0.1g are shown to support at least 20N. Finally, the adhesion method is validated on three robotic monitoring applications: sensor placement, UAV perching, and a climbing robot. These tests demonstrate the utility of biodegradable gelatin hydrogels as adhesives for robotic monitoring applications in natural and industrial settings.

% not only sensor placement, but also longer-term env monitoring in both forest and idsutrial/infrastructure environment. Since sensor placement, perching, climbing, represent robotic monitoring solutions, always need to attach somehow, other solutions are quite specific to substrate, but chemical adhesion works well but leaves residue and not easily reversible. Inspried by hot-melt adhesive, showcase how to use gelatin hydrogel as reversible, bidoegradable adhesive for robobitc montiring in three different use cases, sensor placement, perching drones, and climbing robots. 

The same gelatin hydrogel used as an elastic in the previous work can also function as an adhesive, through heating, pressing against the surface, and cooling. This enables robotic monitoring tasks requiring adhesion across various substrates, such as placing sensors, a perching drone or a climbing robot. These robots can collect data for infrastructure monitoring or industrial inspection or for biodiversity and environmental monitoring in natural environments. In the forest, this can enable access to the different forest strata; sensors can be placed against the trunk fo the tree in the understory, the inner canopy can be reached with a climbing robot, and the emergent layer can be monitoring with a perching UAV (RQ3,4).
Common adhesion methods are specialized for specific substrates, don't generalize to different surfaces, or leave behind harmful residue. Therefore this work presents the novel use of gelatin-based hydrogels as biodegradable and water-soluble adhesives for reversible adhesion to different surfaces for robotic environmental monitoring. The hydrogel adheres through heating, attaching, and cooling. The hydrogel is released by heating again, and any residue can be washed away with water. To correctly dimension the adhesive, factors affecting the maximum pull-off force are experimentally characterized. The versatility of the adhesive is shown through adhesion to different surfaces. Only 0.1g are shown to support at least 20N. The potential applications are then showcased, with sensor placement, UAV perching, and a climbing robot. These tests demonstrate the utility of biodegradable gelatin hydrogels as adhesives for robotic monitoring applications in natural and industrial settings.


\underline{Automation}
The previously described solutions are all manually operated. User-friendliness for the end-user in the field can can be enhanced full or partial automation. For UAVs operating in forest environments, perception and navigation due to the dense foliage which obscures branches are the main challenges. In the following works we look at perception strategies which can be used as building blocks to enable UAV data collection and navigation in forests. This includes multispectral data collection using an event camera and structured light, enabling multispectral data collection from close within the canopy (RQ1). Both methods can be used to assist navigation of UAVs in forest environments, and thus can also assist with automation (RQ2), scalability (RQ3), as well as user-friendliness (RQ2).

\subsubsection{Learning Occluded Branch Depth Maps in Forest Environments Using RGB-D Images \cite{Geckeler2024}}
% %abstract
% Covering over a third of all terrestrial land area, forests are crucial environments; as ecosystems, for farming, and for human leisure. However, they are challenging to access for environmental monitoring, for agricultural uses, and for search and rescue applications. To enter, aerial robots need to fly through dense vegetation, where foliage can be pushed aside, but occluded branches pose critical obstacles. Therefore, we propose pixel-wise depth regression of occluded branches using three different U-Net inspired architectures. Given RGB-D input of trees with partially occluded branches, the models estimate depth values of only the wooden parts of the tree. A large photorealistic simulation dataset comprising around 44Kimages of nine different tree species is generated, on which the models are trained. Extensive evaluation and analysis ofthemodels on this dataset is shown. To improve network generalization to real-world data, different data augmentation and transformation techniques are performed.The approaches are then also successfully demonstrated on real-world data of broadleaf trees from Swiss temperate forests and a tropical Masoala Rain- forest. This work showcases the previously unexplored task of frame-by-frame pixel-based occluded branch depth reconstruction to facilitate robot traversal of forest environments.

As seen in previous works, forests are crucial environments, as ecosystems, for farming, and for human leisure. However, they are challenging to access for environmental monitoring, for agricultural uses, and for search and rescue applications. To enter, aerial robots need to fly through dense vegetation, where foliage can be pushed aside, but occluded branches pose critical obstacles. Therefore, we propose pixel-wise depth regression of occluded branches using three different U-Net inspired architectures. Given RGB-D input of trees with partially occluded branches, the models estimate depth values of only the wooden parts of the tree. A large photorealistic simulation dataset comprising around 44K images of nine different tree species is generated, on which the models are trained. Extensive evaluation and analysis of the models on this dataset is shown. To improve network generalization to real-world data, different data augmentation and transformation techniques are performed. The approaches are then also successfully demonstrated on real-world data of broadleaf trees from Swiss temperate forests and a tropical Masoala Rain- forest. This work showcases the previously unexplored task of frame-by-frame pixel-based occluded branch depth reconstruction to facilitate robot traversal of forest environments.

\subsubsection{WIP: Event Spectroscopy: Event-based Multispectral and Depth Sensing using Structured Light}
%intro, don't have to mention everything (put this in intro to ED)
% A challenge for multispectral and hyperspectral data analysis methods is resolution. While spectral analysis of individual leaves in a lab setting may detect infections and pathogens relatively early, applying these same methods to data collected from dozens of meters above the canopy requires symptoms to be much more visible, with substantial damage incurred before successful detection. Close-range multispectral detections could provide a direct data method for such data, but are challenging to capture on a camera and systems level.


Next, we demonstrate high-fidelity depth and multispectral sensing utilizing an event-camera structured light setup. Multispectral sensing is achieved by projecting different wavelengths as structured light, this multispectral sensing is less dependent on ambient light and can thus be collected closer to the point of interest, such as forest canopies, presenting a new type of multispectral data collection (RQ1). The depth and multispectral sensing is showcased on the downstream task of material differentiation, detecting leaves and branches. The low latency, high-framerate of the event camera makes the setup uniquely suited for dynamic, rapidly changing environments, such as a UAV flying through a forest. Both methods can be used to assist navigation of UAVs in forest environments, and thus can also assist with automation (RQ2), scalability (RQ3), as well as user-friendliness (RQ2).


\subsection{Biodiversity Assessments in Forests}
% besides previously seen enevironmental data, biodiversity assessment is also essenital. In this section we show to use use edna to scalcably assess biodiversity from challenging to access rainforests

% XP represents full pipeline, everythig, all points

Biodiversity assessments in forests are challenging since they are difficult to access and species can be difficult to detect. EDNA presents a method for comprehensive, non-invasive biodiversity assessments. In this section we demonstrate both manual and fully autonomous eDNA collection from surfaces. The methods are validated within the scope of the XPrize Rainforest Competition as part of the winning team ETH BiodivX. During the finals, we successfully demonstrate the full pipeline to monitor 100 ha of rainforest within 24 hours, showcasing the full potential of direct data collection; collecting eDNA from previously difficult to access forest canopies, in a scalable way to cover 100 ha, and user-friendly since it is fully automated.

\subsubsection{eProbe: Sampling of Environmental DNA within Tree Canopies with Drones \cite{Kirchgeorg2024}}
%abstract
%Environmental DNA (eDNA) analysis is a powerful tool for studying biodiversity in forests and tree canopies. However, collecting representative eDNA samples from these high and complex environments remains challenging. Traditional methods, such as surface swabbing or tree rolling, are labor-intensive and require significant effort to achieve adequate coverage. This study proposes a novel approach for unmanned aerial vehicles (UAVs) to collect eDNA within tree canopies by using a surface swabbing technique. The method involves lowering a probe from a hovering UAV into the canopy and collecting eDNA as it descends and ascends through branches and leaves. To achieve this, a custom- designed robotic system was developed featuring a winch and a probe for eDNA collection. The design of the probe was optimized, and a control logic for the winch was developed to reduce the risk of entanglement while ensuring sufficient interaction force to facilitate transfer ofeDNA onto the probe. The effectiveness of this method was demonstrated during the XPRIZE Rainforest Semi-Finals as 10 eDNA samples were collected from the rainforest canopy, and a total of 152 molecular operational taxonomic units (MOTUs) were identified using eDNA metabarcoding. We further investigate how the number of probe interactions with vegetation, the penetration depth, and the sampling duration influence the DNA concentration and community composition of the samples.

\subsubsection{WIP: Robotic Biodiversity Assessments: ETH BiodivX's Robots in the XPRIZE Rainforest Finals}


% Paper Structure (Cyrill): (ie more closely follows the table, then makes all the check boxes at the end
% Chapter 2: manual biodiv/env monitoring in forests
% OG, BOG, SB => Sensor placement + environmental monitoring in new locations (outer-branches forest, near trunk, etc),  all manual

% Chapter 3:
% OBR, ED => towards automation

% Chapter 4: 
% VOL, UR => Sensor placement in agricultural fields + user-friendly (manual)

% Chapter 5:
% EP2, XP => scalable, automated, user-friendly biodv monitoring through eDNO in forest environments

%%%%%%%%%%%%%%%%%%%%%%%%%%%%%%% OR  (Stefano:)
% Chapter 2: env. monitoring in forests
% OG, BOG, SB => Sensor placement + environmental monitoring in new locations (outer-branches forest, near trunk, etc),  all manual + 
% OBR, ED => towards automation

% Chapter 3: Biodiv. monitoring in forests
% % EP2, XP => scalable, automated, user-friendly biodv monitoring through eDNA in forest environments

% Chapter 4: Pest detection in agriculuture
% VOL, UR => Sensor placement in agricultural fields + user-friendly

%%%%%%%%%%%%%%%%%%%%%%%%%%%%%%%%%%%%%%% OR
%================================This is the one========================================
% VOL, UR; OG,BOG,SB; OBR,ED; EP2, XP
% Chapter 2: Pest detection in agriculuture
% VOL, UR => Sensor placement in agricultural fields + user-friendly (don't show automation, etc)

% Chapter 3: env. monitoring in forests
% OG, BOG, SB => Sensor placement + environmental monitoring in new locations (outer-branches forest, near trunk, etc),  all manual + 
% OBR, ED => towards automation

% Chapter 4: Biodiv. monitoring in forests
% % EP2, XP => scalable, automated, user-friendly biodv monitoring through eDNO in forest environments
%========================================================================


% % % Table according to #1SM
% % \begin{table}
% %     \centering 
% %     \begin{tabular}{r|ccccccccc}
% %             % OG  & BOG & SB & OBR & ED & EP-2 & XP & VOL & UR
% %          &  \cite{Geckeler2022a} & \cite{Geckeler2023b} & SB & \cite{Geckeler2024} & ED & \cite{Kirchgeorg2024} & XP & \cite{Geckeler2023a} & \cite{Geckeler2024a}\\
% %          \hline \hline
% %          New Locations          & X & X & X &   &   & X & X & X &  \\
% %          \hline
% %          User Friendliness      &   & X &   & X &   & X & X &   & X\\
% %          \hline
% %          New Data Modalities    &   &   &   &   & X & X & X & X &  \\
% %          \hline
% %          Automation             &   &   &   & X & X &   & X &   & X\\
% %          \hline
% %          Scalability            &   &   & X &   & X & X & X & X & X\\
% %     \end{tabular}
% %     \caption{Caption}
% %     \label{tab:my_label}
% % \end{table}

% % % Table according to #2,CS (switch EP2/XP with VOL/UR)
% % \begin{table}
% %     \centering 
% %     \begin{tabular}{r|ccccccccc}
% %             % OG  & BOG & SB & OBR & ED & VOL & UR & EP2 & XP
% %          &  \cite{Geckeler2022a} & \cite{Geckeler2023b} & SB & \cite{Geckeler2024} & ED & \cite{Geckeler2023a} & \cite{Geckeler2024a} & \cite{Kirchgeorg2024} & XP \\
% %          \hline \hline
% %          New Locations          & X & X & X &   &   & X &   & X & X\\
% %          \hline
% %          User Friendliness      &   & X &   & X &   &   & X & X & X\\
% %          \hline
% %          New Data Modalities    &   &   &   &   & X & X &   & X & X\\
% %          \hline
% %          Automation             &   &   &   & X & X &   & X &   & X\\
% %          \hline
% %          Scalability            &   &   & X &   & X & X & X & X & X\\
% %     \end{tabular}
% %     \caption{Caption}
% %     \label{tab:my_label}
% % \end{table}

% % Table according to #2,CS (switch EP2/XP with VOL/UR), but with rows switched
% \begin{table}
%     \centering 
%     \begin{tabular}{r|ccccccccc}
%             % OG  & BOG & SB & OBR & ED & VOL & UR & EP2 & XP
%          &  \cite{Geckeler2022a} & \cite{Geckeler2023b} & SB & \cite{Geckeler2024} & ED & \cite{Geckeler2023a} & \cite{Geckeler2024a} & \cite{Kirchgeorg2024} & XP \\
%          \hline \hline
%          New Data Modalities    &   &   &   &   & X & X &   & X & X\\
%          \hline
%          Automation             &   &   &   & X & X &   & X &   & X\\
%          \hline
%          Scalability            &   &   & X &   & X & X & X & X & X\\
%          \hline
%          User Friendliness      &   & X &   & X &   &   & X & X & X\\
%          \hline
%          New Locations          & X & X & X &   &   & X  &   & X & X\\
%     \end{tabular}
%     \caption{Sorting According to table, increasing points addressed}
%     \label{tab:my_label}
% \end{table}

% % Table according to #3,CS (but with VOL/UR first), and with rows switched
% \begin{table}
%     \centering 
%     \begin{tabular}{r|cc|ccccc|cc}
%             % VOL & UR & OG  & BOG & SB & OBR & ED & EP2 & XP
%         & \multicolumn{2}{c}{Pest Det.} & \multicolumn{5}{c}{Env. Mon.} & \multicolumn{2}{c}{Biodiv. Assess.}  \\
%          % & Pest Det. & Pest Det. & Env. Mon. & Env. Mon. & Env. Mon. & Env. Mon. & Env. Mon. & Biodiv. Assess. & Biodiv. Assess. \\%TODO: make multicolumn for each of the papers
%          & \cite{Geckeler2023a} & \cite{Geckeler2024a} &  \cite{Geckeler2022a} & \cite{Geckeler2023b} & SB & \cite{Geckeler2024} & ED & \cite{Kirchgeorg2024} & XP \\
%          \hline \hline
%          New Data Types (Q1)  & X &   &   &   &   &   & X & X & X\\
%          % maybe novel data modalities, novel direct data types
%          \hline
%          Automation (Q2)           &   & X &   &   &   & X & X &   & X\\
%          \hline
%          Scalability (Q3)          & X & X &   &   & X &   & X &  X & X\\
%          \hline
%          User Friendliness (Q2)    &   & X &   & X &   & X &   &  X & X\\
%          \hline
%          New Locations (Q4)        & X &   & X & X & X &   &   &  X & X\\
%     \end{tabular}
%     \caption{Sorting according to themes (Pest detection, env. monitoring, biodiveristy assessment)}
%     \label{tab:my_label}
% \end{table}

%Q1: What data is needed for direct data collection? [eDNA, volatiles, direct sensor data]
%%alternative: How to collect direct data for different fields (agriculture, biodiv/env.)
%Q2: How to democratize direct data collection? [automation, user-friendliness]
%Q3: How to scale direct data collection? [use UAV to access new locations + automate ]
%Q4: How to collect direct data from challenging locations? [UAVs for new locations]
