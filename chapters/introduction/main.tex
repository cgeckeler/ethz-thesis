\chapter{Introduction}
\label{ch:introduction}

% \dictum[Immanuel Kant]{%
%   Sapere aude! Habe Mut, dich deines eigenen Verstandes zu bedienen! }%
% \vskip 1em

% \begin{otherlanguage}{ngerman}
% Die ältesten Bestimmungen der wahren Grösse der Moleküle hat die kinetische
% Theorie der Gase ermöglicht, während die an Flüssigkeiten beobachteten
% physikalischen Phänomene bis jetzt zur Bestimmung der Molekülgrössen nicht
% gedient haben. \dots
% \end{otherlanguage}


- Context: biodiversity + climate crisis, data collection is necessary to make informed decisions and to evaluate effectiveness of measures

- Problem: Current data collection methods are inadequate, do not enable right type of data collection, cannot access locations, or do not cover temporal or spatial scales needed for proper data analysis. 

- Goal: Democratizing scalable data collection methods for environmental monitoring, biodiversity assessment, and agriculture for all locations and all users.

- Contributions:
Collect novel (relevant, vs remote sensing) types of data of environmental factors (eDNA, volatiles)
Initial steps to provide easy-to-use solutions for end-users through automation and user-centric design 
Enable novel use cases for UAVs, placing and collecting sensors from tree canopies, or placing and collecting sensors from agricultural fields
Provides access to previously inaccessible locations, and the use of automatable drones permits scaling to larger scales


% Paper Structure (Cyrill):
% Chapter 2: manual biodiv/env monitoring in forests
% OG, BOG, SB => Sensor placement + environmental monitoring in new locations (outer-branches forest, near trunk, etc),  all manual

% Chapter 3:
% OBR, ED => towards automation

% Chapter 4: 
% VOL, UR => Sensor placement in agricultural fields + user-friendly

% Chapter 5:
% EP2, XP => scalable, automated, user-friendly biodv monitoring through eDNO in forest environments

%%%%%%%%%%%%%%%%%%%%%%%%%%%%%%% OR  (Stefano:)
% Chapter 2: env. monitoring in forests
% OG, BOG, SB => Sensor placement + environmental monitoring in new locations (outer-branches forest, near trunk, etc),  all manual + 
% OBR, ED => towards automation

% Chapter 3: Biodiv. monitoring in forests
% % EP2, XP => scalable, automated, user-friendly biodv monitoring through eDNO in forest environments

% Chapter 4: Pest detection in agriculuture
% VOL, UR => Sensor placement in agricultural fields + user-friendly